 Para asegurar la autonomía de los agentes robóticos es necesario proporcionar un sistema
de alimentación mediante el uso de baterías, para ello los agentes robóticos Pololu 3Pi+
 tienen como fuente de energía cuatro celdas de 
 baterías recargables NiMH en serie. Sin embargo, no presentan un sistema de
 carga incorporado debido a esto es necesaria la extracción de las mismas para 
 su recarga. Implicando esto, la participación activa del operador
 para la recarga de las baterías, lo cual a su vez no es escalable  al emplear una gran cantidad de
 agentes robóticos Pololu 3Pi+.

Este trabajo pretende llevar a un mayor grado la autonomía de los agentes Pololu 3Pi+ mediante
la implementación de un sistema de carga multiquímica que permita la recarga del banco de baterías
NiMH incorporado, sin la necesidad de su extracción. Adicionalmente, se incorpora una batería de 
respaldo LiON 18650 para aumentar su autonomía y que pueda ser empleada para la alimentación de módulos
adicionales del agente robótico. Como parte del sistema de carga, también se diseñarán estaciones de carga
magnéticas, de forma que cuando los agentes Pololu 3Pi+ presenten un bajo nivel de carga, estos puedan acudir 
a la estación más cercana para realizar la carga de ambas baterías. 

