En la actualidad se observa el uso extendido de baterías recargables
en sistemas portátiles, así como un incremento en la capacidad de almacenamiento
de energía debido a la creación de nuevas tecnologías para baterías, como lo es 
el caso de las baterías de iones de litio (Li-ion). Las baterías con tecnología 
de níquel metalhidruro (NiMH) también son baterías con un uso extendido en la 
actualidad debido a su antigüedad, sin embargo, presentan una menor densidad energética
en comparación con las batería Li-ion.

En el presente protocolo se hace una propuesta para aumentar la autonomía
de los agentes robóticos Pololu 3Pi+ que son utilizados en el ecosistema Robotat
en la Universidad del Valle de Guatemala. Para aumentar la autonomía de estos 
agentes robóticos se propone la adición de una batería Li-ion como fuente de respaldo
para las baterías NiMH que posee actualmente. Con la adición de la batería de respaldo,
también se propone la integración de un sistema de carga multiquímica para realizar la carga
de ambas baterías junto con el desarrollo de estaciones de carga que se encontraran alrededor 
de la plataforma, de forma que los agentes robóticos puedan ir a recargar las baterías al
momento que presenten un bajo nivel de carga.