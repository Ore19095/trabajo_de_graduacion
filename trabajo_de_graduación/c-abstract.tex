This project focuses on the development of a multi-chemistry charging system for
NiMH and Li-ion batteries. To achieve this, it was necessary to design a 
constant current/voltage source controlled by an analog voltage signal.

To measure the current delivered to the batteries (when operating in constant
current mode), a sensor has been implemented using the TL082 operational 
amplifier in a differential amplifier configuration, along with a $15 \text{m}\Omega$ shunt resistor for current-to-voltage conversion.

Furthermore, a digital-to-analog converter (DAC) has been designed using one of the two operational amplifiers in the TL082, configured as a voltage follower, in conjunction with an RC filter with a cutoff frequency of 72.3 Hz to filter out the harmonics of a PWM signal, allowing only the DC component of the PWM signal to pass through.

For charging NiMH batteries, the charge algorithm known as "negative voltage change"  has been implemented. This algorithm involves detecting a negative change in battery voltage, indicating that the battery is fully charged. For Li-ion battery charging, the "constant current-constant voltage" charge algorithm has been implemented, with the charging process stopping when the current delivered to the battery drops below 10% of the battery's capacity.

A power multiplexer has been implemented to switch the power source from NiMH to Li-ion batteries to the charging system, enabling one battery to be charged at a time. Additionally, a power multiplexer has been implemented to determine which battery is connected to the 3pi+ robot.

The charging system includes a station with female USB A connectors to supply 12V to the Pololu 3Pi+ robot's expansion board and allow the passage of UART communication lines from the microcontroller via a magnetic USB cable.