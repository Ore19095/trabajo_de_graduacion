% necesito redactar la introduccion de mi tesis

En el ámbito de la robótica, y más específicamente en el campo de
 la robótica de enjambres, la autonomía de los agentes robóticos emerge como
un elemento fundamental. La mejora de esta autonomía requiere que estos agentes
estén equipados con sistemas de alimentación que les permitan operar durante 
períodos prolongados sin intervención humana. Para alcanzar este objetivo, es 
necesario que estos agentes dispongan de sistemas de carga que les permitan 
recargar sus baterías de manera autónoma.

En el caso de los agentes robóticos Pololu 3Pi+, se caracterizan por contar
con un banco de baterías NiMH compuesto por 4 celdas en serie, que son 
recargables. No obstante, carecen de un sistema de carga incorporado, lo 
que obliga a extraer las baterías para su recarga.

El presente trabajo se enfoca en diseño y construcción de un sistema 
de carga multiquímica diseñado específicamente para las baterías NiMH y 
Li-ion utilizadas en los agentes robóticos Pololu 3Pi+. Este sistema de carga 
tiene como propósito habilitar la recarga del banco de baterías NiMH sin 
requerir la extracción de las mismas. Además, se introduce una batería de 
respaldo Li-ion 18650 con el fin de ampliar la autonomía de los agentes 
robóticos y brindar energía adicional para alimentar módulos auxiliares.

Como componente esencial del sistema de carga, se han desarrollado estaciones
de carga magnéticas. Estas estaciones permiten que los agentes Pololu 3Pi+ 
identifiquen niveles bajos de carga y acudan automáticamente a la estación más cercana para recargar ambas baterías.
