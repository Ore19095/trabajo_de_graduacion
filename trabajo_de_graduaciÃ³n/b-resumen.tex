Este proyecto se centra en el desarrollo de un sistema de carga multiquímica para baterías NiMH y Li-ion.
Para ello fue necesario el diseño de  una fuente de corriente/voltaje constante 
la cual es  controlada mediante una señal analógica de voltaje.

Para medir la corriente entregada a las baterías (cuando se opera en modo de
corriente constante), se implementó un sensor utilizando el amplificador
operacional TL082 en configuración amplificador diferencial, 
junto con una resistencia shunt de $15 \text{m}\Omega$ para la conversión de corriente a
voltaje.

Además, se diseñó un conversor digital a analógico que utiliza
uno de los dos amplificadores operacionales del TL082 en configuración seguidor
en conjunto con un filtro RC con una frecuencia de corte de 72.3 Hz para filtrar
los armónicos de una señal PWM, de forma que únicamente se tenga a la salida la
componente DC de la señal PWM.

Para la carga de las baterías NiMH se ha implementado el algoritmo de carga 
denominado cambio negativo de voltaje el cual consiste en la detección de un
cambio negativo en el voltaje de la batería, lo cual indica que la batería 
se encuentra completamente cargada. Para la carga de las baterías 
Li-ion se ha implementado el algoritmo de carga
denominado carga por corriente constante-voltaje constante, deteniendo el proceso
de carga cuando la corriente entregada a la batería es menor al 10\% de la capacidad
de la batería.


Se diseñó un multiplexor de potencia con el propósito de gestionar
la carga de las baterías NiMH y
Li-ion, de forma que ambas baterías puedan ser recargadas con un único convertidor
reductor. Además, se ha incorporado otro multiplexor 
de potencia para determinar que batería está proporcionando potencia al 
agente robótico Pololu 3pi+.
 
Adicionalmente se desarrolló una estación de carga que cuenta con conectores USB-A
hembra para suministrar 12 V a la placa de expansión del agente robótico Pololu 3Pi+
y adicionalmente se utilizan las líneas D- y D+ de los conectores USB para la
comunicación UART entre el sistema de carga y la estación.


