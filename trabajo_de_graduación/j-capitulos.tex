\chapter{Sistema de carga multiquímica}

Para la carga de la batería Li-ion se escogió el método de carga CC/VC
descrito en la sección \ref{sec:alg_lion}, la máxima corriente de carga
se estableció en $1\text{A}$. Para la carga de la batería NiMH se utilizó
el método de carga de cambio de voltaje negativo descrito en la sección 
\ref{sec:alg_nimh}. Se empleó el mismo convertidor reductor para la carga
de ambas baterías, por lo que no es posible cargar ambas al mismo tiempo.
Para determinar que batería será cargada se utilizó un multiplexor de potencia
construido con transistores MOSFET de canal P.


\section{Convertidor reductor}

\label{sec:buck_design}

Para el sistema de carga multiquímica es necesario poder proporcionar una 
corriente constante para la carga de ambos tipos de baterías. Para ello
se decidió emplear un convertidor reductor, el cual tiene por componente principal
el circuito integrado LM2596. Para poder realizar un control del voltaje de salida ( y de
forma indirecta la corriente del mismo)
de forma electrónica, es decir aplicando un voltaje, se realizó una modificación al circuito
de realimentación del circuito como es mostrado en la figura \ref{fig:buck_modificado}.

\begin{figure}[H]
    \centering
    \includegraphics[scale=0.35]{imagenes/buck_control_simplificado.png}
    \caption{Convertidor reductor con realimentación modificación (versión simplificada) }
    \label{fig:buck_modificado}
\end{figure}

Para este convertidor se definieron los siguientes parámetros de rendimiento:

$$ \Delta I = 150 \text{mA} $$
$$ \Delta V = 1 \text{mV}$$

Otros parámetros útiles para el diseño de este convertidor serán el voltaje
de alimentación ($V_g$), así como la frecuencia de conmutación, $f_s$ (dada por la hoja 
de datos del fabricante), los valores son:

$$V_g = 12\text{V}$$
$$ f_s = \frac{1}{T_s} = 150 \text{KHz}$$



\subsection{Cálculo de resistencias para el circuito de retroalimentación}

\label{sec:res_ret}

Para poder determinar en que forma va a variar el voltaje en el nodo $V_{out}$ se realizó un 
análisis de nodos en el nodo FB correspondiente a la figura \ref*{fig:buck_modificado}, 
obteniendo la ecuación \ref*{eq:buck_control}.

\begin{equation}
    V_{FB} = \frac{R_1R_2}{R_1R_2+R_3(R_1+R_2)} V_{out} + \frac{R_1R_3}{R_1R_3+R_2(R_1+R_3)} V_2
    \label{eq:buck_control}
\end{equation}

Donde $V_2$ el el voltaje aplicado de forma externa para modificar el valor
de voltaje a la salida. De la ecuación \ref*{eq:buck_control} se puede 
observar que el valor de $V_{FB}$ es dependiente tanto de $V_2$ como de
$V_{out}$. 

Para determinar los valores de las resistencias se armó un sistema de dos ecuaciones,
fijando los valores de $V_{FB}$,$V_{out}$,$V_2$, y $R_1$. Los dos valores necesarios
de $V{out}$ se obtuvieron fijando el valor mínimo y máximo que podría tener a la salida
el convertidor, los cuales fueron de $2.8\text{V}$ y $6.4\text{V}$ de forma que pueda usarse tanto para 
cargar la celda de batería Li-ion, así como las 4 celdas NiMH. El valor de $V_{FB}$ 
es el voltaje de referencia del IC LM2596. Los valores del voltaje de control se
establecieron de forma que, cuando $V_2 = 5\text{V}$ el voltaje a la salida sea de 
$2.8\text{V}$ mientras que cuando  $V_2 = 0\text{V}$ el voltaje sea de $6.4\text{V}$.
Por último, se escogió un valor de $750\Omega$ para $R_1$ de forma arbitraria, siguiendo
la recomendación de la hoja de datos del LM2596.

Con lo explicado anteriormente, el sistema de ecuaciones a resolver es el siguiente:
\begin{eqnarray}
     \frac{750R_3}{ 750R_3 + R_2(750 + R_3)} 6.4\text{V} = 1.23\text{V} \\
    \frac{750R_3}{ 750R_3 + R_2(750 + R_3)} 2.8\text{V} + \frac{750R_2}{750R_2+R_3(750+R_2)} 5\text{V} = 1.23\text{V}   
\end{eqnarray}
dando como resultado los siguientes valores para $R_2$ y $R_3$:

$$R_2 = 2596.64\Omega$$ 
$$R_3 = 3606.44\Omega$$

Puesto que estos valores no son comerciales, se emplearon valores de resistencias en 
paralelo, de forma que sea posible aproximar estos valores. Para el valor de $R_2$ 
se empleó una resistencia de $3\text{K}\Omega$ y $20\text{K}\Omega$ dando como 
resistencia equivalente $2.608\text{K}\Omega$. De igual forma se utilizó una 
resistencia de $4.7\text{K}\Omega$ en paralelo con una de $15\text{K}\Omega$ para aproximar
 el valor de $R_3$, dando una resistencia total de $3.578\text{K}\Omega$.

Al reemplazar los valores para $R_2$ y $R_3$ utilizados en la ecuación \ref{eq:buck_control}
asi como un valor de $V_2 = 5\text{V}$ se obtiene que el valor de $V_{out}$ es de $2.786\text{V}$
mientras que si se reemplaza con $V_2 = 0\text{V}$ se obtiene que el valor de $V_{out}$ es de
$6.431\text{V}$, los cuales son valores aceptables para la carga de ambas baterías,
de forma que la variación en las resistencias no afecta significativamente
el rango de operación del convertidor.

\subsection{Componentes externos del convertidor}

Los componentes externos necesarios para que el convertidor esté completo, son el 
diodo $D_1$, el inductor $L_1$, y el capacitor $C_1$ que se muestran en la figura
\ref{fig:buck_modificado}.

 Para $D_1$ se escogió el diodo 1N5817, el cual es un diodo \textit{
 schotky}, la elección de este componente fue debido a su baja caída de voltaje
  ($0.45\text{V}$ al conducir $1\text{A}$) lo cual minimiza las pérdidas por 
  conducción del mismo, mejorando la eficiencia del convertidor.

Ya que el valor del inductor es dependiente
del valor del voltaje a la salida del convertidor es necesario determinar cuál
sería el valor óptimo. Para ello se definió la función $L(V)$ a partir de las 
ecuaciones \ref{eq:ripple_L} y \ref{eq:salida_buck}. La función es la siguiente: 

\begin{equation}
    L(V) = \frac{V_g - V}{2\Delta I}\frac{V}{V_g} T_s
    \label{eq:L_function}
\end{equation}

Posteriormente se obtuvo que el valor en
donde la de derivada de $L(V)$ tiene un valor de cero es en $V = \frac{V_g}{2}$,
por lo tanto, el valor máximo que tomará $L(V)$ será de 

$$ L =  \frac{T_sV_g}{8\Delta I}$$
 
Reemplazando los valores en la ecuación anterior se obtiene que el valor de 
$L_1$ para el cual se asegura como máximo un rizado de $150 \text{mA}$ en 
todo el rango de operación, es de $66.6 \mu \text{H}$, por lo que se usó un
inductor de $68 \mu \text{H}$. Este aumento en el valor del inductor no afecta
de manera negativa el funcionamiento del convertidor, ya que el valor calculado
anteriormente unicamente es un límite inferior para el valor del inductor, un 
valor mas alto reducirá a un mas el rizado en la corriente de salida, lo que 
mejora el funcionamiento del convertidor.

Por último, se empleó la ecuación \ref{eq:ripple_C} para obtener el valor 
de $C_1$, siendo este de $125 \mu\text{F}$, por lo que se utilizó un capacitor
electrolítico de $220 \mu\text{F}$ debido a que es el siguiente valor comercial 
más alto disponible. De igual forma que con el inductor, el
aumento en el valor del capacitor no afecta de manera negativa el funcionamiento
del convertidor, ya que el valor calculado anteriormente es un límite inferior
para el valor del capacitor, un valor más alto reducirá a un más el rizado en
el voltaje de salida.

\subsection{Componentes adicionales}

Para asegurar un funcionamiento óptimo en \cite{lm2596}, se aconseja incorporar
un capacitor en paralelo con la resistencia $R_2$ mostrada en la figura
\ref{fig:buck_modificado}. La elección del valor de este condensador se basó
en las pautas proporcionadas en el cuadro \ref{tb:feedforward_cap}. 
Dado que el voltaje de salida mínimo es de $2.8\text{V}$, se optó por utilizar
un capacitor de $33\text{nF}$.

\begin{table}[t]
    \centering
    \begin{tabular}{|c|c|}
        \hline
        Voltaje de salida (V) & Capacitor \\
        \hline
        2 & $33\text{nF}$ \\
        4 & $10\text{nF}$ \\
        6 & $3.3\text{nF}$ \\
        9 & $1.5\text{nF}$ \\
        12 & $1\text{nF}$ \\
        15 & $680\text{pF}$ \\
        24 & $220\text{pF}$ \\
        28 & $220\text{pF}$ \\
        \hline
    \end{tabular}
    \caption{Valores para capacitancia de retroalimentación según \cite{lm2596}
        con respecto al voltaje de salida.}
    \label{tb:feedforward_cap}
\end{table}



Otro componente adicional necesario es un capacitor de desacople a la entrada
del LM2596 para mejorar la estabilidad del voltaje de alimentación. Para su
selección se empleó como criterio el uso de la figura \ref{fig:rms_cap}. Según
\cite{lm2596} es necesario asegurar que el capacitor de salida pueda tolerar
una corriente RMS entre el $50\%$ y $75\%$ de la corriente de salida.
Para dar un mayor margen, se decidió escoger un capacitor que soporte una corriente
RMS igual a la corriente máxima esperada, por lo que se escogió un capacitor
de $470\mu\text{F}$ a $25\text{V}$.

\begin{figure}[t]
    \centering

    \includegraphics[scale=.4]{imagenes/rms_cap.png}
    \caption{capacidad de corriente RMS}
    \label{fig:rms_cap}

\end{figure}

Con todo lo discutido anteriormente, el diseño del convertidor reductor es mostrado
en la figura \ref{fig:buck_finalizado}.

\begin{figure}[H]
    \centering

    \includegraphics[scale=.45]{imagenes/buck_control_finalizado.png}
    \caption{Diseño del convertidor reductor finalizado}
    \label{fig:buck_finalizado}
\end{figure}

\subsection{Simulación del convertidor diseñado}

Para comprobar el correcto funcionamiento se realizó una simulación del 
convertidor diseñado. Para ello se empleó el simulador LTspice, junto con 
el modelo spice proporcionado por texas instruments para el circuito integrado
LM2596 (descarga disponible en \cite{noauthor_lm2596_nodate}). 

\begin{figure}[H]
    \centering
    \includegraphics[scale=0.35]{imagenes/resultado_simulacion.png}
    \caption{Respuesta del convertidor (azul) contra el voltaje de control
            (negro).}
    \label{fig:sim_buck}
\end{figure}

En la figura \ref{fig:sim_buck} se muestra la respuesta del convertidor al
aplicar una señal escalón de $5\text{V}$ en la entrada de control. Se puede
observar que cuando el valor en el voltaje de control es de $0\text{V}$
la salida es de $6.408\text{V}$, además
cuando el valor del voltaje de control es de $5\text{V}$ el voltaje a la salida
del convertidor es de $2.768\text{V}$. Los cuales son valores cercanos a los esperados
al momento de realizar el diseño. 

Adicionalmente se puede observar que el rizado en el voltaje de salida del 
convertidor no es observable a la escala de voltaje en el que se está, esto debido
al uso de un valor más alto al calculado tanto para el inductor como para el 
capacitor del convertidor.

                                                                
\subsection{Prueba física del circuito}

Como último paso se realizó el diseño de un circuito impreso para probar el 
funcionamiento del circuito, en la figura \ref{fig:buck_pcb} se muestra la capa
superior e inferior del PCB.

\begin{figure}[H]
    \centering
    \begin{subfigure}{0.45\linewidth}
        \centering
        \includegraphics[scale=0.3]{imagenes/top_buck.png}
        \caption{Capa superior}
    \end{subfigure}
    \begin{subfigure}{0.45\linewidth}
        \centering
        \includegraphics[scale=0.3]{imagenes/bottom_buck.png}
        \caption{Capa inferior}
    \end{subfigure}
    \vfill
    \begin{subfigure}{0.45\linewidth}
        \centering
        \includegraphics[scale=0.45]{imagenes/3d_buck.png}
        \caption{Vista 3D}
    \end{subfigure}
    \begin{subfigure}{0.45\linewidth}
        \centering
        \includegraphics[scale=0.12]{imagenes/circuit_phis.jpg}
        \caption{PCB ensamblada}
    \end{subfigure}
    \caption{Diseño del PCB para el convertidor reductor}
    \label{fig:buck_pcb}
\end{figure}

Debido a que se manejan corrientes de hasta $1.5\text{A}$ es necesario que el
diseño del PCB sea capaz de manejar dicha corriente, para ello se empleó una 
calculadora de ancho de pistas \cite{noauthor_pcb_nodate}, con el cual se 
determinó que el ancho de las pistas debe ser de $0.53\text{mm}$ como mínimo,
esto tomando en cuenta que se estará usando un cobre con espesor de 1oz, y que
el maximo aumento de temperatura permitido es de 10\textordmasculine C. 

Se programó el microcontrolador ATmega328p para que genere una señal 
cuadrada de $5\text{V}$ con un periodo de 10 segundos en el pin PB1,
la cual se conectó a la entrada de control del convertidor. Al medir el voltaje
en la salida se obtuvo que el voltaje cuando se tienen $5\text{V}$ en la entrada
de control es de $2.76\text{V}$, mientras que cuando se tiene $0\text{V}$
en la entrada de control el voltaje a la salida es de $6.40\text{V}$, estos 
valores son cercanos a los esperados, por lo que el convertidor diseñado
funciona de forma correcta.

El circuito construido en protoboard para la prueba del convertidor se muestra
en la figura \ref{fig:prueba_buck}.

\begin{figure}[H]	
    \centering
    \includegraphics[scale=0.2]{imagenes/prueba_buck.jpg}
    \caption{Circuito para prueba del convertidor reductor (Arduino Nano únicamente
    es para programar el ATmega328p)} 
    \label{fig:prueba_buck}
\end{figure}

El codigo empleado en la prueba se muestra a continuación.

    \begin{lstlisting}

        #define F_CPU 8000000UL
        #include <avr/io.h>
        #include <util/delay.h>


        int main(void) {
                // ----     IO CONFIG  ---------
            DDRB = 0b00000011; // PB[7:2] input, PB1 out, PB0 out
            
            PORTB = 0b0000000; //activar PB0
            while (1) {
                PORTB |= 1<<1; //PB1 en 1
                _delay_ms(10000);
                PORTB &= ~(1<<1); //PB1 en 0
                _delay_ms(10000);
            }
        }

    \end{lstlisting}






\section{Sensor de corriente}

    Para poder usar el convertidor reductor explicado en la sección
    \ref{sec:buck_design} es necesario poder medir la corriente que se
    está entregando a la batería. Para ello se diseñó un sensor de corriente
    basado en el circuito integrado TL082, el cual es un amplificador operacional
    de propósito general.

    El sensor tiene 2 componentes pricipales, el amplificador diferencial y 
    la resistencia de sensado. El amplificador diferencial se encarga de 
    amplificar la diferencia de potencial entre los terminales de la resistencia
    de sensado, la cual es proporcional a la corriente que circula por la misma.
    La resistencia de sensado tiene un valor de $15 \text{m}\Omega$, con una tolerancia
    de $\pm 5\%$.

    Para poder medir una corriente máxima de $1.5\text{A}$ se escogió que la ganancia
    para amplificador diferencial (ver figura \ref{fig:ampDif}) sea de $200\text{V/V}$, por lo que la salida del
    sensor será de $4.5\text{V}$ cuando la corriente sea de $1.5\text{A}$. Para obtener
    los valores de resistencias necesarios para obtener la ganancia deseada se
    empleó la ecuación \ref{eq:ampDifSimp}, fijando el valor de $R_2$ en $1.5\text{K}\Omega$,
    obteniendo que el valor de $R_4$ debe ser de $300\text{K}\Omega$. Siguiento con la
    simplificación del circuito se obtiene que el valor de $R_1$ debe ser de $1.5\text{K}\Omega$
    y el valor de $R_3$ debe ser de $300\text{K}\Omega$. El sensor de corriente diseñado
    se muestra en la figura \ref{fig:sensor_corriente}.

    \begin{figure}[H]
        \centering
        \includegraphics[scale=0.35]{imagenes/current_sensor.png}
        \caption{Sensor de corriente diseñado}
        \label{fig:sensor_corriente}
    \end{figure}

    con lo explicado anteriormente, el sensor tiene una ganancia final de 
    $3\text{V/I}$, la ecuación \ref{eq:sensor} describe la relación entre
    la corriente que circula por la resistencia de sensado y el voltaje de salida
    del amplificador diferencial.

    \begin{equation}
        V_{out} = 3I_1
        \label{eq:sensor}
    \end{equation}
    
    Para la alimentación del sensor es necesario un mínimo de $\pm7\text{V}$ en
    los pines de alimentación del amplificador operacional TL082, esto por dos 
    motivos, el primero es que el voltaje de salida del sensor es de $4.5\text{V}$
    cuando la corriente es de $1.5\text{A}$ tomando encuenta que este opamp 
    no es del tipo \textit{rail to rail}, por lo que el voltaje de salida no
    puede ser igual al voltaje de alimentación, y el segundo motivo es que el
    voltaje en modo común de este amplificador tiene como limite superior el
    voltaje de alimentación positivo, por lo que es necesario que el voltaje 
    de alimentación sea mayor que el voltaje maximo de que recibirá en sus entradas,
    que para este caso es de $6.4\text{V}$. Para alimentar el sensor se empleó
    una alimentación simétrica de $\pm 12\text{V}$.

    
    \subsection{Mediciones del Sensor de corriente}

    Para comprobar el correcto funcionamiento del sensor de corriente diseñado
    se realizaron mediciones para varios niveles de corriente. Para ello se
    empleó una fuente de voltaje simetrica a $\pm7\text{V}$, y se utilizaron resistencias de entre 
    $4\Omega$ y $30\Omega$ para simular la carga de la batería. Los resultados 
    obtenidos son mostrados en el cuadro \ref{tb:mediciones_sensor}. 

\begin{table}[H]
    \centering
    \begin{tabular}{|c|c|c|c|}
        \hline
    Corriente medida & Salida del sensor (V) & Valor Ideal (V) & Error (\%) \\
    \hline
    0.241            & 0.815                 & 0.795           & 2.48       \\
    0.480            & 1.546                 & 1584            & 2.40       \\
    0.713            & 2.270                 & 2.353           & 3.52       \\
    0.933            & 2.945                 & 3.079           & 4.35       \\
    1.143            & 3.592                 & 3.772           & 4.77       \\
    1.354            & 4.24                  & 4.47            & 5.11       \\
    1.564            & 4.88                  & 5.16            & 5.43 \\
     \hline     
    \end{tabular}

    \caption{Valores de salida del sensor de corriente diseñado}
    \label{tb:mediciones_sensor}
    \end{table}


    En la figura \ref{fig:mediciones_sensor} se muestra el circuito construido
    en protoboard para la medición de los valores presentados anteriormente.

    \begin{figure}[H]
        \centering
        \includegraphics[scale=0.2]{imagenes/prueba_sensor.jpg}
        \caption{Circuito para medición del sensor de corriente}
        \label{fig:mediciones_sensor}
    \end{figure}


\section{Conversor digital a analógico (DAC)}

    Para controlar el voltaje de salida del convertidor reductor es necesario
    poder generar un voltaje de control de forma digital. Para ello se diseñó
    un conversor digital a analógico (DAC) utilizando un filtro RC de primer
    orden y un amplificador operacional en modo seguidor de voltaje. El DAC
    propuesto se muestra en la figura \ref{fig:dac}.

    \begin{figure}[H]
        \centering
        \includegraphics[scale=0.5]{imagenes/dac_disenio.png}
        \caption{DAC propuesto}
        \label{fig:dac}
    \end{figure}

    Ya que para la generación de la de PWM se utilizó el módulo PWM1 (debido
    a que este es el que presenta una mayor resolución) del ATmega328p, con el 
    cual se puede generar una señal PWM con una resolución de 10 bits, que en
    conjunto con una frecuencia de operación para el microcontrolador de $8\text{MHz}$
    se obtiene una frecuencia de PWM de $7.81\text{KHz}$, de determinó que la 
    frecuencia de corte del filtro RC sea de $7\text{Hz}$, de forma el armonico
    de mayor magnitud (el cual es el de frecuencia $7.81\text{KHz}$) sea atenuado
    en $-40\text{dB}$. 

    La frecuencia de corte para el filtro RC está dada por la ecuación \ref{eq:fc_dac},
    donde $R$ es el valor de la resistencia y $C$ es el valor del capacitor, mostrados
    en la figura \ref{fig:dac}.

    \begin{equation}
        f_c = \frac{1}{2\pi RC}
        \label{eq:fc_dac}
    \end{equation}

    Se fijó el valor del capacitor en $1\mu\text{F}$, por lo que el valor de la
    resistencia es de $22.7\text{K}\Omega$, por lo que se utilizará el valor 
    de resistencia comercial más cercano, el cual es de $22\text{K}\Omega$.


\section{Controlador}

    Para la gestión de los multiplexores de potencia y el control del convertidor
    reductor se diseñó un controlador basado en el microcontrolador ATmega328p.
    El cual tiene las siguientes funciones:

    \begin{itemize}
        \item Controlar el voltaje y corriente de salida del convertidor reductor
        \item Controlar que batería se está cargando mediante un multiplexor de potencia
        \item Controlar el multiplexor de potencia para seleccionar la batería
        que proporcionará la alimentación al agente robótico.
        \item Permitir la comunicación con la estación de carga mediante comunicación
        UART.
        \item  Permitir la comunicación con el adaptador wifi mediante comunicación
        I2C.
        \item indicar de forma visual el estado de carga de las baterías mediante
        2 LEDs.
    \end{itemize}


    \section{Placa de expansión para el agente robótico}

        Luego de haber probado todos los sistemas en conjunto se procedió a diseñar
        una placa de expansión para el agente robótico Pololu 3Pi+ que permita
        la conexión de todos los sistemas diseñados. Para permitir la conexión
         entre ambas placas se emplaron conectores 
        macho ubicados según los planos proporcionados por la empresa Pololu \cite{noauthor_pololu_nodate}.
        Los pines utilizados para la comunicación o transmisión de potencia entre
        ambas placas son los siguientes:

        \begin{itemize}
            \item GND: Todos los pines etiquetados como GND en la placa de control
            del agente robotico son conectados con la tierra de la placa de expansión.
            \item VBAT: Este pin es utilizado para realizar la carga de las baterías
            NiMH presentes en el agente robótico.
            \item VSW: Este pin es utilizado para alimentar alimentar la placa de 
            control del agente robótico, es utilizado para conectar la batería Li-ion
            al agente robótico, al momento de que la batería NiMH se encuentre descargada.

        \end{itemize}

    
        


\chapter{Estación de carga}

La estación de carga esta compuesta por 2 partes, la primera es un circuito impreso
en el cual se encuentran los conectores USB, para la transmisión de potencia y 
datos entre la estación de carga y el sistema de carga multiquímica. La segunda
parte es una estructura impresa en 3D, la cual tiene como función principal
proveer un espacio para colocar el circuito impreso descrito anteriormente y 
los cables USB magneticos que se utilizarán para realizar la conexión entre
la estación de carga y el sistema de carga multiquímica.


\section{Circuito impreso}

El circuito impreso tiene 4 componentes principales que se listan a continuación:

\begin{itemize}
    \item Conectores USB tipo A hembra
    \item Cables USB para transmisión de potencia y datos
    \item \textit{Headers} macho para comunicación serial
    \item Bornera de 2 pines para alimentación de la estación de carga
\end{itemize}
El circuito impreso diseñado se muestra en la figura \ref{fig:pcb_estacion_carga}.

\begin{figure}[H]
    \centering
    \begin{subfigure}{0.9\linewidth}
        \centering
        \includegraphics[scale=0.25]{imagenes/bottom_charging.png}
        \caption{Capa inferior}
    \end{subfigure}
    \vfill
    \begin{subfigure}{0.9\linewidth}
        \centering
        \includegraphics[scale=0.45]{imagenes/3d_charging.png}
        \caption{Vista 3D}
    \end{subfigure}
    \caption{Diseño del PCB para la estación de carga}
    \label{fig:pcb_estacion_carga}
\end{figure}

La PCB de la estación de carga soldada se muestra en la figura \ref{fig:pcb_estacion_carga_soldada}.

\begin{figure}[H]
    \centering
    \includegraphics[scale=0.2]{imagenes/charger_pcb.jpg}
    \caption{PCB de la estación de carga soldada}
    \label{fig:pcb_estacion_carga_soldada}
\end{figure}