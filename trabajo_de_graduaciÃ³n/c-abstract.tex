his project focuses on the development of a multi-chemistry charging system 
for NiMH and Li-ion batteries. To achieve this, it was necessary to design a
 constant current/voltage source that is controlled by an analog voltage signal.

To measure the current delivered to the batteries (when operating in constant
current mode), a sensor has been implemented using the TL082 operational 
amplifier in a differential amplifier configuration, along with a 
$15 \text{m}\Omega$ shunt resistor for current-to-voltage conversion.

Additionally, a digital-to-analog converter has been designed, utilizing
one of the two operational amplifiers of the TL082 in a follower configuration,
along with an RC filter with a cutoff frequency of 72.3 Hz to filter out the 
harmonics of a PWM signal so that only the DC component of the PWM signal is 
present at the output.

For the NiMH battery charging, the negative voltage change charging algorithm 
has been implemented, which involves detecting a negative change in the battery
voltage, indicating that the battery is fully charged. For Li-ion battery 
charging, the constant current-constant voltage charging algorithm has been
implemented, halting the charging process when the current delivered to the 
battery is less than 10\% of the battery's capacity.

A power multiplexer was designed to manage the charging of both NiMH and
Li-ion batteries, allowing both batteries to be recharged with a single 
buck converter. Additionally, another power multiplexer has been incorporated 
to determine which battery is providing power to the Pololu 3pi+ robotic agent.

Furthermore, a charging station was developed, equipped with female USB-A 
connectors to supply 12 V to the expansion board of the Pololu 3Pi+ robotic 
agent. Additionally, the D- and D+ lines of the USB connectors are used for 
UART communication between the charging system and the station.