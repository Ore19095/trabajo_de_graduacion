
\begin{itemize}
    \item Se diseñó un convertidor DC-DC reductor basado en el circuito
    integrado LM2596, capaz de entregar una corriente de hasta 1.5A, y
    con la capacidad de regular el voltaje de salida entre $2.7$V y 
    $6.4$V mediante una señal analógica.

    %\item Se implementaron algoritmos para la carga de baterías de Li-ion y
    %NiMH, lo cual incluye la detección de inicio y fin de carga de cada
    %batería.

    \item Se logró diseñar una placa de expansión para el agente robótico
    Pololu 3Pi+ que incorpora el sistema de carga multiquímica capaz de
    gestionar la recarga de baterías de Li-ion y NiMH,
    así como poder realizar la conmutación mediante un multiplexor de 
    potencia, entre ambas baterías para la 
    alimentación del agente robótico.

    \item Se realizó el diseño y la construcción de una estación de carga
    que proporciona tanto el voltaje como la corriente requerida para el
    funcionamiento óptimo del sistema de carga multiquímica, asi como la
    comunicación con el sistema de carga mediante comunicación UART.

    \item Se diseñó un sensor de corriente y convertidor digital a 
    analógico con el amplificador operacional TL082, con el fin de
    medir la corriente entregada por el convertidor DC-DC a las baterías y 
    proporcionar una salida analógica para el control del convertidor DC-DC.


\end{itemize}