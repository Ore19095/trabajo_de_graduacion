\subsection*{Investigación preliminar}

Lo primero a realizar para este proyecto es una Investigación sobre distintos
métodos de carga tanto para baterías Li-ion así como para baterías NiMH. Adicionalmente
se investigarán circuitos integrados que cumplan estas funciones así como circuitos
discretos, es decir circuitos que no se encuentran integrados en un encapsulado, para 
la carga de las baterías.  Así también, se realizará una investigación sobre la construcción
de multiplexores de potencia, tanto con circuitos integrados existentes como con componentes
discretos.

\subsection*{Simulación del circuito para la carga de baterías Li-ion y NiMH}

Para este paso, luego de haber seleccionado los circuitos para la carga de baterías
Li-ion y NiMH, se realizará su simulación para determinar distintos parámetros, como lo 
es la potencia consumida por cada componente, su corriente y voltaje entre otros.
Esto para poder dimensionar de forma correcta cada componente al momento de realizar 
el diseño del circuito impreso(PCB).

\subsection*{Prototipado de los circuitos para la carga de baterías}

Se realizará el diseño y manufactura del circuito impreso para cada uno
de los circuitos de carga, de forma independiente uno del otro, para luego
comprobar su correcto funcionamiento así como determinar el tiempo de carga
de cada batería.

\subsection*{Simulación del multiplexor de potencia}

Se realizará la simulación del circuito seleccionado para realizar la función
de multiplexado para la selección de la fuente de alimentación (batería Li-ion
o NiMH) para el Pololu 3Pi+. También se verificará su comportamiento tanto
en estado estacionario como de su comportamiento en el régimen transitorio.

\subsection*{Prototipado del multiplexor de potencia}

Para este paso se llevará a cabo el diseño y manufactura del circuito impreso
para el multiplexor de potencia, verificando su comportamiento con relación a 
la simulación. Se realizarán mediciones para verificar que el comportamiento 
en el régimen transitorio sea el adecuado para proveer potencia de forma 
ininterrumpida al agente robótico Pololu 3Pi+.

\subsection*{Integración de los circuitos diseñados}

Se realizará la integración de los circuitos diseñados previamente en un 
único circuito impreso, para luego verificar que cada uno funcione de forma
correcta cada uno, así como también su operación en conjunto. Se verificará
que sea posible la carga de las dos baterías al mismo tiempo mientras se
emplea la fuente de alimentación externa para el suministro de potencia del
Pololu 3Pi+.

\subsection*{Implementación del algoritmo para detección de carga en las baterías}

Se procederá a implementar un algoritmo para determinar el estado de carga de baterías,
de forma que en el momento en el cual una de las dos baterías tenga un estado de carga bajo,
se cambie a la batería de respaldo, o en caso de que ambas se encuentren con un estado de 
carga bajo, iniciar el proceso para que el agente robótico Pololu 3Pi+ se dirija hacia su 
estación de carga.

\subsection*{Desarrollo del prototipo para la estación de carga}

Como ultima actividad se realizará el desarrollo del prototipo para la estación de carga. 
Este desarrollo comprenderá tanto la estructura física para la estación como el \textit{hardware}
para poder proveer de potencia al sistema de carga multiquímica así como para detectar que la 
estación se encuentra en uso. 



