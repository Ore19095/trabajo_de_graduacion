\section{Controlador}
    \label{sec:controlador}
    Para la gestión de los multiplexores de potencia y el control del convertidor
    reductor se diseñó un controlador basado en el microcontrolador ATmega328p.
    El cual tiene las siguientes funciones:

    \begin{itemize}
        \item Controlar el voltaje y corriente de salida del convertidor reductor
        \item Controlar que batería se está cargando mediante un multiplexor de potencia
        \item Controlar el multiplexor de potencia para seleccionar la batería
        que proporcionará la alimentación al agente robótico.
        \item indicar de forma visual el estado de carga de las baterías mediante
        2 LEDs.
    \end{itemize}

    Para el control de la corriente y voltaje de salida del convertidor reductor
    se empleó un controlador PID para el voltaje y un controlador PID para la corriente.
    Para la obtención de un aproximado de cada uno de los parámetros se emplearon 
    los datos obtenidos de la simulación del convertidor reductor, con los cuales 
    se obtuvo la funcion de transferencia del sistema tanto para el voltaje como
    para la corriente, asumiendo que la carga es una resistencia de 10$\Omega$. Con
    la planta del sistema fue utilizado el PID tuner de MATLAB para obtener los
    parámetros del controlador PID. Estos fueron calibrados para que funcionaran 
    de forma correcta con el convertidor reductor, los valores finales de los
    parámetros del controlador PID se muestran en el cuadro \ref{tab:pid_parameters}.

    \begin{table}[H]
        \centering
        \begin{tabular}{|c|c|c|}
            \hline
            Parámetro & Voltaje & Corriente\\
            \hline
            $K_p$ & -1.2515 & -0.1252\\
            \hline
            $K_i$ & -500.5 & -40.05 \\
            \hline
            $K_d$ & -2.244e-4 & 8.488e-5 \\
            \hline
        \end{tabular}
        \caption{Parámetros del controlador PID}
        \label{tab:pid_parameters}
    \end{table}

    Para el control de los multiplexores de potencia se emplearon los pines
    PB0 y PD7 para seleccionar la batería que se cargará, y los pines PB2 y 
    PD4 para seleccionar la batería que alimentará al agente robótico.

    El pin PD5 se empleó para en control de la activación y desactivación
    del convertidor reductor. Para indicar el estado de carga de las baterías
    se emplearon los pines PB6 y PB7 los cuales tienen un led conectado a cada
    uno de ellos.

    Con respecto a los algoritmos de carga para ambas baterías, en el caso
    de la batería LiON se empleó el algoritmo de carga CC-CV, descrito en la
    sección \ref{sec:alg_lion}. Para la batería de NiMH se empleó el
    algoritmo de carga denominado \textit{trickle charge}, descrito también en
    la sección \ref{sec:alg_nimh}.

    Para determinar cuando una batería se encuentra descargada en ambos casos es por medio
de la medición de la tensión de la batería. En el caso de la batería LiON se considera
que la batería se encuentra descargada cuando la tensión de la batería es menor a 3V,
mientras que para la batería de NiMH se considera que la batería se encuentra descargada
al momento de que su tensión es menor o igual a 4V.




